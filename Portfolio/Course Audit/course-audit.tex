\documentclass[9pt,twocolumn,oneside]{amsart}	%defines this as an article
\usepackage{chrisfriend-comp} %provides formatting declarations for page, headers, figures, textcolor, comments, and bibliographic styles
\usepackage{chrisfriend-OTF-support} %provides support for OTF system fonts; incompatible with latex, rtf2latex, & ht4latex
%\usepackage[utf8]{inputenc} %support for smallamp?

%\usepackage{draftwatermark}

%\usepackage{tabularx}
\usepackage{tabulary} % allows for the tables I make rubrics with
%\usepackage{supertabular}
\usepackage{xtab} % allows tables to span pages
\usepackage{booktabs} % allows fancy lines in tables
\usepackage{rotating} % allows landscape tables
\usepackage{lscape} % allows rotated longtables
\usepackage{multirow} % allows rowspanning
\usepackage{enumitem} % helps with the overview
%\usepackage{paralist}
\usepackage{multicol}

\title[Course Audit]{Assignment Sheet: Course Audit}
\chead{\scriptsize{\MakeUppercase{Course Audit}}}

\begin{document}
%\bibliographystyle{abbrv}
\thispagestyle{empty}
\setlength{\columnsep}{.25in}

\twocolumn[
%\vspace{-2in}
\begin{center}
\huge
{\includegraphics[height=1.75\baselineskip]{pegasus.pdf}}

\textbf{Assignment Sheet: Course Audit}

{\normalsize Chris Friend • \textsc{enc1101} • Spring 2014}
\end{center}
\vspace{1.5\baselineskip}
] %Use for column-spanning the title

%\begin{multicols}{2}
\section{Background and Purpose} % (fold)
\label{sec:background}
Throughout this course, you have been struggling with new perspectives on writing: how writing is done, how writing is used, and how writing functions. For this assignment, you will examine that struggle and show how it worked. Consider what you have learned both explicitly through your reading and intrinsically through your work. What did you figure out in that 3:00 a.m.\ Red Bull\textregistered{}-induced craze? What have you learned or figured out this semester, even if it wasn't expected (by you or me)?

Step back from your assignments and look at the ``big picture'' of the course. Review the Course Syllabus to recall how your assignments were designed to fit together. Then, write Friend a letter \textbf{asserting how well you achieved the course outcomes.} The letter format affords more casual language than a scholarly essay, but it constrains you to a smaller number of pages. Consider what you think is worth emphasizing about your experiences this term.%\footnote{I plan to spend a few minutes in class brainstorming other affordances and constraints for the letter format. I'd also like to get my students to come up with either an organizational structure or an include/exclude list for content.}
% section background (end)

This document serves as the cover letter for your final portfolio. Below your Course Audit, create a Table of Contents with links to your revised essays. Consider using the affordances of the medium: create linked text within your Audit pointing to documents as you discuss them, or to a specific section/sentence of another document (check out \texttt{Insert\ $\rightarrow$\ Bookmark}). The more you help me find the evidence you're providing, the easier it is for me to recognize your awesomeness. And that is, after all, the point\ldots{}right?

\begin{table}[b]
	\caption{Evaluation Rubric for Course Audit}\label{tab:rubric}
	\small
\begin{tabulary}{\columnwidth}{cLLL}
	\toprule  & \textbf{\textsc{Outcomes}} & \textbf{\textsc{Identity}}\\
\midrule	\textbf{Excellent} 
& Connects outcomes to growth as writer, effectively uses coursework for support 
& Relates work on assignments to development of thinking as a writer or student \\
%& Evaluates how writer's work on research process connects with writing abilities or academic progress \\
\midrule	\textbf{Adequate} 
& Demonstrates (w/ ex's) achievement of course outcomes 
& Shows awareness that assignments connect \\
%& Illustrates understanding of research process as one of genuine inquiry \\
\midrule	\textbf{Poor} 
& Fails to address course outcomes 
& Omits review of own skillset \\
%& Omits discussion of research process \\
	\bottomrule
\end{tabulary}
\end{table}
% section rubric (end)
\newpage

%\newpage
\section{Procedure} % (fold)
\label{sec:procedure}
\begin{description}
	\item[Brainstorm] Review the course outcomes listed on the syllabus. Which of the outcomes have you accomplished this semester? What did you do this semester that proves you met those goals?
	\item[Organize] Find specific examples of each course outcome in the material you have created for this class.
	\item[Write] Draft a letter that explains how you met the outcomes and refers to the examples you came up with. This letter will be read by your instructor and possibly by a program assessment committee from the Department of Writing and Rhetoric. In other words, your audience is familiar with the expectations and terminology used in Comp I at \textsc{ucf}. You should refer to the expectations and use the terminology, rather than taking time to define them.
	\item[Revise] Double-check your document organization and goals. Make sure you address each outcome and support yourself with specific, convincing evidence. If you claimed that this course or your instructor has completely changed your life or made you the Bestest. Writer. Evar\texttrademark{}, please revise. This document is not designed to inflate my ego or earn you an A. Instead, it exists to show me where to find evidence that you've achieved the course's goals. Stay focused.
	\item[Compile] Assemble all supporting documents into a portfolio that argues your case. (An \href{https://docs.google.com/document/d/1n3Pm7OlCYKaHclO51UotlhfNRfp5Zz5fynGeNpKABvI/edit?usp=sharing}{audit/portfolio template} is available on Google Docs; \href{https://webcourses.ucf.edu/courses/1009451/files#ENC1101_CMB-14Spring%2FFinal%20Portfolio%2FStudent%20Samples%20from%201102}{sample Course Audits} are available on Webcourses.) At minimum, include revised versions of three (3) major assignments: Process Analysis, Science Accommodation, and one discourse community paper. If additional discourse community papers or even minor assignments help illustrate that you achieved course outcomes, feel free to include them.
%	\item[Include] This Course Audit is part of your Final Portfolio. Include it as a form of cover sheet, then add the material you compiled in the previous step.
%	\item[Relax] Smile, knowing there's nothing else you have to do for this class.
\end{description}

\section{Evaluation} % (fold)
\label{sec:rubric}
%This assignment is not designed to evaluate your ability to kiss up to your instructor and claim that the class has immeasurably changed your life for the better. 
The focus of the assignment is to get you to take a broad view of the content and goals of the course. If you are able to address the course content and comment on your experiences within it, you will have done your job. The rubric included in Table~\ref{tab:rubric} provides evaluation details.

% section procedure (end)



% section rubric (end)
%\end{multicols}
\end{document}
