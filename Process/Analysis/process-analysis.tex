\documentclass[10pt, oneside]{amsart}	%defines this as an article
\usepackage{chrisfriend-comp} %provides formatting declarations for page, headers, figures, textcolor, comments, and bibliographic styles
\usepackage{chrisfriend-OTF-support} %provides support for OTF system fonts; incompatible with latex, rtf2latex, & ht4latex
%\usepackage[utf8]{inputenc} %support for smallamp?

%\usepackage{tabularx}
\usepackage{tabulary} % allows for the tables I make rubrics with
%\usepackage{supertabular}
\usepackage{xtab} % allows tables to span pages
\usepackage{booktabs} % allows fancy lines in tables
%\usepackage{rotating} % allows landscape tables
\usepackage{lscape} % allows rotated longtables
\usepackage{multirow} % allows rowspanning
\usepackage{enumitem} % helps with the overview
%\usepackage{paralist}
\usepackage{draftwatermark}

\title[Process Analysis]{Assignment Sheet: Process Analysis}
\chead{\scriptsize{Assignment Sheet: Process Analysis}}

  
\begin{document}
%\bibliographystyle{abbrv}
\thispagestyle{empty}
\vspace{-2in}
\begin{center}
\huge
\includegraphics[height=1.75\baselineskip]{pegasus.pdf}

\textbf{Assignment Sheet: Process Analysis}

{\normalsize Chris Friend • \textsc{enc1101} • Spring 2014}
\end{center}
\vspace{1.5\baselineskip}

\section{Background} % (fold)
\label{sec:background}
If the writing process takes place exclusively inside the head of the writer, how can composition researchers learn how other people write? A common method used by composition researchers is a “think-aloud” protocol in which an author is asked to say everything that comes to mind while writing. Like Sondra Perl, you will observe a writer's process to see what you can learn. Unlike Perl, you will be your own research subject by studying your own “think-aloud” from your previous assignment.

You started the semester reading from John Swales about the moves made by authors when creating introductions to their research. The paper will report your findings from original research, so you should incorporate those moves, as well. Start with what is already known, establish your niche, and present your findings. (Make a map, find the gap, and fill the gap.) Follow the traditional \textsc{imrd} pattern for the remainder of your report.

Additionally, we will employ Lamott's approach to “shitty first drafts” by conducting extensive peer reviews in class. You should also strive to avoid Rose's “rigid rules” to help your writing flow more naturally. As you can see, this paper will document and illustrate your understanding of the readings we have done.
% section background (end)

\begin{comment}
	\section{Purpose} % (fold)
	\label{sec:purpose}
	\begin{itemize}
		\item To better understand yourself as a writer
		\item To relate your writing process with the processes of others we have read about
		\item To practice writing in an academic genre
		\item To use the \textsc{cars} and \textsc{imrd} models in your writing, where appropriate
		\item To understand research as a process of genuine inquiry
	\end{itemize}
	% section purpose (end)
\end{comment}

\section{Procedure} % (fold)
\label{sec:procedure}
More discussion of each of the stages of this project can be found on \textsc{waw} 322–25, and student examples from Clayton Stark (\textsc{waw} 278–91) and Dominieq Ransom (\textsc{waw} 292–98) show two approaches others have taken.\footnote{As you read the student samples, remember that they are “final” drafts submitted with a portfolio, and that Stark was in an Honors course. Your papers are still experimental; theirs are polished.} The general stages of the process can be summarized as follows:
\begin{description}
	\item[Brainstorming] What kinds of things would you like to learn about your own writing process?
	\item[Researching] You will use the “Portrait of a Writer” assignment as your testing scenario. Record yourself writing that paper so that you can analyze your process in that assignment for use in this one.
	\item[Analyzing] Transcribe the recording you made all of your writing process.	We will work in class to build a code that can be used to analyze that transcript. With highlighters, codes, and creative thinking, you will identify trends and events in your writing process.%\newpage
	\item[Planning] Determine the following parameters for what you choose to write:
	\begin{itemize}
		\item Your intended audience (Who could benefit from your newfound knowledge?)
		\item Your purpose for writing (What are you trying to accomplish with this paper?)
		\item Your level of formality (How can you sound academic when writing about yourself?)
	\end{itemize}
	\item[Drafting] Be sure to include both the \textsc{cars} and the \textsc{imrd} steps to mimic the structure of a scholarly research report.
	\item[Revising] Be prepared to make substantial revisions to your first draft. Refer to the rubric to make sure you meet the expectations you have for your own grade.
\end{description}% section procedure (end)

\section{Assessment} % (fold)
\label{sec:assessment}
Because your instructor has only known you for a short while, this assignment clearly cannot be graded in terms of accuracy—there is no way to verify whether what you identify as your writing process is correct. Instead, what you need to do in your paper is show the thinking that you have done while trying to create it. What have you observed while doing your research? What have you learned about yourself while doing this project? What might you rely on more or even do differently as a result of what you have seen?

By approaching this paper as an opportunity to share your growth and development as a writer and your learning and experiences as a researcher, you can create a paper that is interesting, informative, and personally valuable. To assign the grade that your paper earns, your instructor will evaluate how clearly you identify and answer your research question, how well you show your awareness of your writing process, how thoroughly you reflect on what you have learned, and the clarity with which you compose the paper. (See Table~\ref{tab:rubric} for details.) Using the \textsc{cars} and \textsc{imrd} steps will help ensure that your writing is clear and professionally organized.

\begin{table}[b]
	\tablehead{}
\begin{tabular}{>{\bfseries}cp{1.25in}p{1.25in}p{1.25in}p{1.25in}}
\toprule & \textbf{\textsc{Question}}\newline (15 pts) & \textbf{\textsc{Awareness}}\newline (15 pts) & \textbf{\textsc{Reflection}}\newline (15 pts) & \textbf{\textsc{Clarity}}\newline (5 pts)\\
	\midrule
A &
Provides an insightful answer to the stated research question. &
Writer expresses meta-awareness of growth or change as a result of research. &
Author analyzes the research conclusions and applies them to future situations. &
Writing is clear and easy to follow/understand.
\\
\midrule
C &
Identifies and attempts to answer a research question. &
Writer shows awareness of the writing process as seen in primary research. &
Author reflects on the experience and draws conclusions about the writing process. &
Writing may have occasional issues that do not severely detract from readability.
\\
\midrule
F &
Provides no clear research question, perhaps resorting to a thesis statement. &
Writer does not show awareness of the writing process found in research. &
Author presents research findings but fails to reflect on them or draw connections. &
Technical problems make the writing difficult to understand and distract from its ideas.
\\
	\bottomrule\\
\end{tabular}
\caption{Process Analysis Grading Rubric}\label{tab:rubric}
\end{table}
% section assessment (end)

\end{document}

\section{Formatting} % (fold)
\label{sec:formatting}
You are expected to use \textsc{mla} formatting conventions for this assignment. Another template is available from Webcourses to simplify the process of preparing your document, but regardless of how you create your document, be sure to include:
\begin{itemize}
	\item double-spaced lines,
	\item one-inch margins on all sides and half-inch indents for paragraphs,
	\item a 12-point typeface with serifs (like Times New Roman, \emph{not} Calibri), and
	\item parenthetical citations and a Works Cited page.
\end{itemize}
% section formatting (end)

\section{Deadline} % (fold)
\label{sec:deadline}
This assignment is due by the beginning of class on \textbf{Wednesday, 14 Sept 2011}.
% section deadline (end)
