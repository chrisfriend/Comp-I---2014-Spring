\documentclass[10pt, oneside]{amsart}	%defines this as an article
\usepackage{chrisfriend-comp} %provides formatting declarations for page, headers, figures, textcolor, comments, and bibliographic styles
\usepackage{chrisfriend-OTF-support} %provides support for OTF system fonts; incompatible with latex, rtf2latex, & ht4latex
%\usepackage[utf8]{inputenc} %support for smallamp?

%\usepackage{tabularx}
\usepackage{tabulary} % allows for the tables I make rubrics with
%\usepackage{supertabular}
\usepackage{xtab} % allows tables to span pages
\usepackage{booktabs} % allows fancy lines in tables
%\usepackage{rotating} % allows landscape tables
\usepackage{lscape} % allows rotated longtables
\usepackage{multirow} % allows rowspanning
\usepackage{enumitem} % helps with the overview
%\usepackage{paralist}
\usepackage{multicol}
\usepackage{fullpage}
%\usepackage{draftwatermark}

\title[Portrait of a Writer]{Assignment Sheet: Portrait of a Writer}
\chead{\scriptsize{Portrait of a Writer}}

  
\begin{document}
%\bibliographystyle{abbrv}

\vspace{-2in}
\begin{center}
\huge
\includegraphics[height=1.75\baselineskip]{pegasus.pdf}

\textbf{Assignment Sheet: Portrait of a Writer}

{\normalsize Chris Friend • \textsc{enc1101} • Spring 2014}
\end{center}
\vspace{1\baselineskip}

\begin{multicols}{2}
\thispagestyle{empty}
	\section{Background and Purpose} % (fold)
	\label{sec:purpose}
	You have started to explore academic writing and have seen how Sondra Perl studied her students. Using a similar process, you will study \emph{yourself} to see what you can learn about your own writing habits. To do that, you must first observe yourself while you write. This assignment serves as a personal introduction and as material for you to study later.
	% section purpose (end)

	\section{Procedure} % (fold)
	\label{sec:procedure}
Before you write your self-portrait:
		\begin{enumerate}
			\item Document your writing environment in detail. Use as many senses as possible, and describe elements that may not seem immediately relevant.
			\item Set up and test your recording equipment. Use video, if possible. If not, audio will suffice.\footnote{It is up to you to choose a suitable recording method. Sometimes the simplest solutions are the best. On a Mac, I recommend using QuickTime Player. Selecting \texttt{New Movie Recording} from the \texttt{File} menu is all you need. On a PC, you can use Windows Movie Maker, but resist the urge to edit. Simpler software is better here.}
		\end{enumerate}
Once you are ready to write, begin recording yourself and complete these steps:
	\begin{enumerate}
		\item Compose two pages of text that describe you as a writer. Consider the following questions as potential prompts, but do not make the document a Q\&A session:
		\begin{itemize}
			\item What is your writing history like? What kinds of writing have you done?
			\item What about writing do you enjoy? despise?
			\item How do you write? What procedures or rituals do you have? What do you leave to chance?
			\item How much pre-writing do you do? Do you compose everything in your head before you start, or do you develop your ideas as you go?
			\item What are your reactions to your own writing?
		\end{itemize}
		\item After composing the first draft, perform your typical initial edits or revisions. Keep the camera rolling.
		\item When your document is “complete‚” submit it to Webcourses as an attachment. Note: \textbf{do not} submit your video. That's for you to keep and use later.
	\end{enumerate}
	% section procedure (end)

	\section{Assessment} % (fold)
	\label{sec:assessment}
	Because this assignment exists to give you something to record, it will be assessed for completion, not quality. I will use your submission to get a sense of your writing style, writing ability, and self-perception. Will I notice the kinds of ``errors'' that Tony was constantly trying to correct? Probably. I'm an English teacher. Will I take off points for them? No. Not at all; that's not the point of this assignment. To show you the way I typically assess major papers, a \emph{hypothetical} grading rubric is presented in Table~\ref{tab:rubric}, but remember that it will not be used on this assignment.

\begin{comment}
	\section{Formatting} % (fold)
	\label{sec:formatting}
	Because this is not a formal paper for this course, the formatting expectations are very lax. I request that the body text be in a double-spaced, 12-point typeface with serifs. Times New Roman works fine. One-inch margins on all sides will start a habit that will help when \textsc{mla} formatting becomes a requirement. A boring template (in Word format) is available on Webcourses, if you don't want to worry about formatting at all (because—let's face it—the content is what matters).
	% section formatting (end)
\end{comment}

\end{multicols}



\begin{table}[h]
	{\small
\caption{Hypothetical Grading Rubric}\label{tab:rubric}
	\tablehead{}
\begin{tabular}{>{\bfseries}cp{1.4in}p{1.4in}p{1.4in}p{1.4in}}
\toprule & \textbf{\textsc{Style}} & \textbf{\textsc{Clarity}} & \textbf{\textsc{Structure}} & \textbf{\textsc{Self-Perception}}\\
	\midrule
A &
Draft has strong individual style, distinct from that of peers. Could only be written by you. &
Draft is easy to read and understand. No issues muddle the ideas being expressed. &
Sentences and ¶s are arranged logically; descriptions flow naturally together. &
You understand your writing process as a component of your development as a writer.
\\
\midrule
C &
Draft has slight sense of style. Personal and interesting, but not remarkable. &
Draft is generally easy to read. Few technical errors exist, but they do not distract. &
Sentence-level organization may occasionally be rough, but ¶s are logically organized. &
You show evidence that you are aware of your own processes and thinking.
\\
\midrule
F &
Draft is lacking a sense of style. Text is dry. Writing is notable for being formulaic. &
Mechanical errors make drafts difficult to comprehend and detract from content. &
No organizational structure is evident; ideas wander rather than progress. &
Draft includes no recognition of writing process or fails to present a portrait.
\\
	\bottomrule
\end{tabular}
} %end small
\end{table}
% section assessment (end)


\end{document}

\section{Deadline} % (fold)
\label{sec:deadline}
This assignment is due by the beginning of class on \textbf{Monday, 29 Aug 2011}.
% section deadline (end)
