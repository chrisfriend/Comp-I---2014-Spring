\documentclass[10pt, twosides]{amsart}	%defines this as an article
\usepackage{chrisfriend-comp} %provides formatting declarations for page, headers, figures, textcolor, comments, and bibliographic styles
\usepackage{chrisfriend-OTF-support} %provides support for OTF system fonts; incompatible with latex, rtf2latex, & ht4latex
%\usepackage[utf8]{inputenc} %support for smallamp?

%\usepackage{tabularx}
\usepackage{tabulary} % allows for the tables I make rubrics with
%\usepackage{supertabular}
\usepackage{xtab} % allows tables to span pages
\usepackage{booktabs} % allows fancy lines in tables
\usepackage{rotating} % allows landscape tables
\usepackage{lscape} % allows rotated longtables
\usepackage{multirow} % allows rowspanning
\usepackage{enumitem} % helps with the overview
%\usepackage{paralist}
\usepackage{draftwatermark}

\title[Navigating Sources That Disagree]{Assignment Sheet: Navigating Sources That Disagree}
\chead{\scriptsize{\MakeUppercase{Navigating Sources That Disagree}}}

\begin{document}
%\bibliographystyle{abbrv}

\vspace{-2in}
\begin{center}
\huge
\includegraphics[height=1.75\baselineskip]{pegasus.pdf}

\textbf{Assignment Sheet: Navigating Sources That Disagree}


{\normalsize Adapted from Wardle and Downs' \emph{Writing About Writing}, 165–67.

Chris Friend • \textsc{enc1101} • Spring 2014}
\end{center}
\vspace{1.5\baselineskip}
\thispagestyle{empty}

\section{Background} % (fold)
\label{sec:background}
In your last paper (Analysis of Science Accommodation), you examined multiple sources that reported on the same information. You found different explanations, but no disagreement. This time, your task is to find multiple sources writing about a single arguable issue, but taking different sides. Your sources need to disagree. We will be examining how authors position themselves and their writing, and how they use writing to achieve a goal (see Haas and Flower, \textsc{waw} p. 125, ¶12.).

The articles you found for your previous paper had similar basic goals---to inform their audiences about a discovery. The biggest difference in the presentations was the intended audience and that audience's expectations. To prepare for this paper, you will find articles that intentionally differ in their \emph{purposes}, but the \emph{audiences} may not be so distinct.

For the Science Accommodation paper, the more blatantly different the presentations were, the easier the paper was for you to write. For this assignment, the more \emph{subtle} the differences, the easier your job becomes. You are tasked with answering this question: \textbf{How do authors on disagreeing sides of a published issue situate their arguments?} To find the answer, look at the moves made by the authors, not just the points they make. If your authors are simply stating a yes/no or agree/disagree issue, there's little to analyze. But if you find authors who have to carefully justify or explain their stance, you have much more rhetoric to examine.

Because this is your final major paper for the semester, you are expected to demonstrate a good deal of familiarity with concepts discussed throughout the term. Overall, your paper should show that you can do the following:
\begin{itemize}
	\item Explain how readers and writers use writing as a tool for negotiation.
	\item Illustrate that meaning is socially constructed.
%	\item Understand how texts are constructed.
	\item Investigate the rhetorical situations in which writing exists.
	\item Synthesize multiple sources of information into a coherent argument.
	\item Demonstrate facility with the terms \emph{exigence}, \emph{rhetor}, \emph{construct} (noun and verb), \emph{rhetorical situation}, \emph{claim} vs. \emph{argument}, and \emph{constraint}.
\end{itemize}
% section background (end)

\section{Procedure} % (fold)
\label{sec:procedure}
For this assignment, the \textsc{waw} text has additional details about the assignment (165–67) and a sample student paper (156–164). This time, the sample paper was written by a student in a single semester at \textsc{ucf} in the same \textsc{enc1101} course you are in. I recommend reading Talbot's paper for an excellent example of this assignment.

That said, the process of building this paper breaks down into these general steps:
\begin{enumerate}
	\item \textbf{Find a topic.} Your topic must have multiple sides to it; there must be disagreement. That disagreement should be murky and subtle; black-and-white or clear cases are harder to analyze. The discussion must take place in a published forum. Written communication is best; if you choose a spoken-word forum, you must have transcripts of the discussion.
\begin{comment}
While not required, the following ideas might help you find a topic of practical interest:
		\begin{itemize}
			\item Consider politics. No, I'm serious. People never stop talking (or arguing) about political issues, and opinions are rarely a case of right/wrong or yes/no, despite the appearances of our two-party system.
			\item Avoid religion. People never stop talking about this, either, but because religious beliefs are based on assumptions that are fundamental to how people think, it is nearly impossible to focus exclusively on the argument. In other words, don't go there.
			\item Be interested, but not entrenched. The topic should of course be one you want to study in detail, but it should not be one that infuriates you when hearing the opposition. Consider this an exercise in open-mindedness.
			\item Consider current \emph{and past} events. You're welcome to study issues facing your community right now if you want a sense of relevance. You're also free to study historic disagreements if your interests lie there.
			\item Check your major. If you want to make this paper relevant to your degree, ask people in the department you're majoring in for advice. Who are two scholars who argue in writing? What contentious issues can they suggest?
		\end{itemize}%\todo[inline]{I plan to make these suggestions available in an online, 3-page version of this document, but hide them from the one I run off so it fits on a single piece of paper.}
\end{comment}
%	\end{description}
	\item \textbf{Find the contestants.} You must have three sources discussing the same issue, and those sources must disagree. The murkier and more subtle that disagreement, the better. Blatant arguing is tough to analyze.
	\item \textbf{Analyze the arguments.} First, identify the rhetors' positions: What is at stake for each? How are they related to the issue? Then, identify what values the rhetors assume their audience holds. (You might have done something similar in your Authority paper.) Finally identify the affordances and constraints faced by the authors. How do those shape the writing?
	
	Next, look at the arguments being made by each rhetor. What points do they make in support of their argument? What claims do they maintain? What assumptions do they start from? (This last question is easier to answer if you aren't personally involved. You might want to ask a friend or roommate to help you think through it.)
	\item \textbf{Answer the research question.} Draw conclusions from what you read, then make a claim. Support that claim using evidence from the texts. Remember: you are proving your answer to the question of \emph{how} the authors argue. Do not justify one side/view over the other. In short, don't take sides; analyze. You are not addressing who is right or wrong; you are identifying how the fight is played out.
	\item \textbf{Draft your paper.} Once again, the structure is up to you. (See Webcourses for a suggestion.) Follow the same formatting requirements that have applied to all other papers.
\begin{comment}
	\item \textbf{Draft your paper.} Once again, the structure is up to you. This suggested outline is a starting point:
	\begin{description}
		\item[Background] What articles from composition have you read that give you the exigence to write this paper? Position your writing within the relevant thinking of others. Then, briefly (likely 1¶), orient your reader to the issue you've chosen to explore. What do the articles debate? What sort of context encloses them?
		\item[Claim] Identify the disagreement. What are the perspectives, and what do you claim about the rhetors involved? This is where you answer the research question.
		\item[Proof] Support your claim by providing examples from the articles. You will need to employ an organizational method that makes sense based on your analysis.
		\item[Conclusion] Answer ``So what?'' by explaining what you have determined about the assumptions and making of meaning, as well as the construction of texts. Why are those things good to know? (To answer that last question, you need to know your audience. It might be easiest to answer in a political debate.)
	\end{description}%\todo[inline]{This list would also be hidden or condensed for the printed version.}
\end{comment}
	\item \textbf{Revise your draft.} Using the rubric below, plus the questions on \textsc{waw} 166, consider a visit to the \textsc{uwc} for a fresh perspective.
\end{enumerate}
% section process (end)

\section{Rubric} % (fold)
\label{sec:rubric}
In the last two assignments, your goal was to analyze texts—that is, to explore how they worked and why the authors did what they did. You treated the texts in isolation and held them up as separate examples. In this paper, your goal is to \emph{synthesize}—that is, to make connections between the things you identify. As your build your case for this paper, you will analyze multiple sources and navigate your way through the authors' presentations. From there, you can find common threads and reach conclusions that highlight trends within the disagreement you examine.

\begin{table}[h]
	\caption{Navigating Sources That Disagree Grading Rubric}\label{tab:rubric}
\begin{tabulary}{\textwidth}{cJJJJ}
	\toprule  & \textbf{\textsc{Terms}}\newline (5 pt) & \textbf{\textsc{Rhetorical Situation}}\newline (20 pts) & \textbf{\textsc{Synthesis}}\newline (15 pts) & \textbf{\textsc{Textual Support}}\newline (10 pts)\\
	\midrule \textbf{A} & 
	Natural and sophisticated use of terms from the unit. &%Claim/Sig
	Confidently and adeptly identifies the rhetorical situations surrounding the disagreeing sources. &%Source/Auth
	Connections btwn. sources \& conclusions drawn from analysis are insightful \& clearly presented. &%Lexis/Tone
	Quotes from articles effectively illustrate and support author's claims. %Values
	\\
	\midrule \textbf{C} &
	Accurate use of terms from this unit. &%Claim/Sig
	IDs the rhetorical situations surrounding sources; may not treat them as a conversation; may lack sophistication. &%Source/Auth
	Author draws connections between sources and makes relevant conclusions about rhetoric. &%Lexis/Tone
	Quotes are used consistently but choppily, or they are too infrequent to sufficiently support claims. %Values
	\\
	\midrule \textbf{F} &
	Incorrect or absent use of studied  terms. &%Claim/Sig
	Rhetorical situation not identified or presented as a list, not a discussion. &%Source/Auth
	Paper's conclusion is primarily summary or restatement, or articles are treated in isolation throughout. &%Lexis/Tone
	Quotes are isolated and not related to the claim, nonexistent, or haltingly presented. %Values
	\\
	\bottomrule
\end{tabulary}
\end{table}
% section rubric (end)

\end{document}

\section{Formatting} % (fold)
\label{sec:formatting}
As with the previous assignments, you are expected to use \textsc{mla} formatting conventions, including:
\begin{itemize}
	\item double-spaced lines,
	\item one-inch margins on all sides and half-inch indents for paragraphs,
	\item a 12-point typeface with serifs (like Times New Roman, \emph{not} Calibri), and
	\item parenthetical citations and a Works Cited page.
\end{itemize}
% section formatting (end)
