\documentclass[10pt, oneside, twocolumn]{amsart}	%defines this as an article
\usepackage{chrisfriend-comp} %provides formatting declarations for page, headers, figures, textcolor, comments, and bibliographic styles
\usepackage{chrisfriend-OTF-support} %provides support for OTF system fonts; incompatible with latex, rtf2latex, & ht4latex
%\usepackage[utf8]{inputenc} %support for smallamp?

%\usepackage{tabularx}
\usepackage{tabulary} % allows for the tables I make rubrics with
%\usepackage{supertabular}
\usepackage{xtab} % allows tables to span pages
\usepackage{booktabs} % allows fancy lines in tables
\usepackage{rotating} % allows landscape tables
\usepackage{lscape} % allows rotated longtables
\usepackage{multirow} % allows rowspanning
\usepackage{enumitem} % helps with the overview
\usepackage{acronym}
\input{acronyms}
%\usepackage{paralist}
%\usepackage{draftwatermark}
\usepackage{epigraph}

\title[Genre Analysis]{Assignment Sheet: Genre Analysis}
\chead{\scriptsize{\MakeUppercase{Genre Analysis}}}
\lhead{\scriptsize{\textsc{enc 1101}}}

\begin{document}
%\bibliographystyle{abbrv}
\thispagestyle{empty}

\twocolumn[
%\vspace{-3in}
\begin{center}
\huge
{\includegraphics[height=1.75\baselineskip]{pegasus.pdf}}

\textbf{Assignment Sheet: Genre Analysis}

{\normalsize Chris Friend • \textsc{enc1101} • Spring 2014}
\end{center}
\vspace{\baselineskip}
] %Use for column-spanning the title

\section{Background and Purpose} % (fold)
\label{sec:background}
According to Amy Devitt (1993), ``Genres develop … because they respond appropriately to situations that writers encounter repeatedly. In principle … writers first respond in fitting ways and hence similarly to recurring situations; then, the similarities among those appropriate responses become established as generic conventions. In practice … genres already exist and hence already constrain responses to situations'' (p. 576, emphasis added).\footnote{Devitt, A. (1993). Generalizing about genre: New conceptions of an old concept. \emph{College Composition and Communication}, 44(4):573--86.} 

In your educational experiences, you written in various situations only \emph{after} the ``generic conventions'' have been established by others. The rhetorical situation has been manufactured by teachers or test authors. For your first paper in this class, you were introduced to academic journal articles and told to ``mushfake'' in that genre. Your school experiences have prepared you to start with the genre and write from there.

For this assignment, %we are going to break with that tradition and work against those experiences. instead of figuring out how to fit with an already-established genre, 
you will peel back the conventions to see what's behind. You will look at what \emph{creates} a genre in the first place as you work to answer this question: \textbf{What are the origins, contexts of use, affordances, and constraints of a particular genre?} By analyzing the genre's origins, you will focus on the recurring rhetorical situation that created the genre you are examining.
% section background (end)

\begin{comment}
	\section{Evaluation} % (fold)
	\label{sec:evaluation}
	Your task for this assignment is to identify the characteristics of a genre and its scenes of use. Your work will be assessed using the criteria presented in Table~\ref{tab:rubric}.
	% section evaluation (end)
\end{comment}

\begin{table}[b]
	\caption{Genre Analysis Grading Rubric}\label{tab:rubric}
\begin{tabulary}{\columnwidth}{cJ}
%	\toprule  & \textbf{\textsc{Claims}} & \textbf{\textsc{Authority}} & \textbf{\textsc{Lexis and Tone}}\\ %removed & \textbf{\textsc{Values}}


\toprule \textbf{Great} & \textbf{Evaluates} how the relevant DC uses the genre to achieve its goals\\ %``in the communicative furtherance of its aims'' (Swales, 472)		\\
\midrule \textbf{Good} & \textbf{Illustrates} intertextuality within the genre \\%, drawing from the collected samples as evidence		\\
\midrule \textbf{Okay} & \textbf{Explains}, with evidence from genre samples, %using evidence gathered from genre samples, 
the scene of the genre's use		\\
\midrule \textbf{Poor} & \textbf{States} the situation of the genre's use		\\
%\midrule \textbf{F} & \textbf{Omits} a discussion of the scene leading to the genre's use		\\


	\bottomrule
\end{tabulary}
\end{table}


\section{Required Reading} % (fold)
	\label{sec:readings}
	For this assignment, you will need to read ``Generalizing about Genre: New Conceptions of an Old Concept'' by Amy Devitt, available through Webcourses, plus ``Intertextuality and the Discourse Community'' by James Porter, found on pages 86--96 of your \ac{waw} text.
	% section readings (end)
\vfill
\section{Procedure} % (fold)
\label{sec:procedure}
\noindent Before you write your draft:
		\begin{enumerate}
			\item Select a genre to study based on your in-class brainstorming. What intrigues you enough to study it in depth?
			\item Collect samples of the genre. The more, the merrier, but you need at least three separate samples.
			\item Identify the \textbf{scene} and describe the \textbf{situation} in which the genre is used. Include discussions of the setting, subject, participants, and purpose of your genre. See the Genre Analysis handout for specific questions to help with this analysis.
			\item Identify and describe the genre's \textbf{features}. Refer to the Genre Analysis handout.
			\item Hypothesize what those elements reveal about the genre's scene/situation of use.
		\end{enumerate}
In your paper:
		\begin{enumerate}
			\item Make a claim about what the genre you have chosen tells us about the people who use it and the scene in which it is used.
			\item Support your claim with the evidence you gathered above.
		\end{enumerate}
To make your paper exceptional:
		\begin{enumerate}
			\item Identify, based on your samples, the intertextual characteristics and content specific to this genre.
			\item Evaluate how well the genre you chose meets the needs you identified. Consideration of affordances and constraints will be helpful here.
		\end{enumerate}
% section process (end)


\end{document}

\section{Formatting} % (fold)
\label{sec:formatting}
As with the previous assignments, you are expected to use \textsc{mla} formatting conventions, including:
\begin{itemize}
	\item double-spaced lines,
	\item one-inch margins on all sides and half-inch indents for paragraphs,
	\item a 12-point typeface with serifs (like Times New Roman, \emph{not} Calibri), and
	\item parenthetical citations, where appropriate, and a Works Cited page.
\end{itemize}
% section formatting (end)
