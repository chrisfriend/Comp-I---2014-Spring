%% For help on available commands, see the following documentation files, available at CTAN (http://www.ctan.org/).
%%    usepackage{acronym, babel, biblatex, biblatex-apa, biblatex-apa-test, booktabs, calc, color, comment, csquotes, datetime, docmute, fontspec, footmisc, geometry, graphicx, hyperref, ifxetex, inputenc, isodateo, lineno, longtable, memman, mparhack, multirow, nameref, paralist, siunitx, suffix, svn-multi, todonotes, ulem, xcolor, xltxtra, xtab, xunicode}
%%
%% Available options to the eyerdoc-common style include:
%%	apa		uses APA6-like citations and references (one of apa or mla must be specified)
%%	mla		uses MLA6-like citations and references elements (one of apa or mla must be specified)
%%	nosvn	operates without subversion versian control features
%%  strict	uses strict apa or mla geometry and design elements
%% Additionally, the eyerdoc-common style requires the memoir class and uses its article, draft, and final options
%% For more information about the eyerdoc styles, see http://www.eyer.us/eyerdoc
%%

% Preamble and document properties (fold)
\documentclass[10pt,article,oneside,twocolumn]{memoir}
\usepackage[apa,nosvn]{eyerdoc-common} % for information on eyerdoc packages, see http://www.eyer.us/eyerdoc
\usepackage{eyerdoc-chrisfriend-fonts}
\bibliography{bibliography}
\input{acronyms}
\usepackage{epigraph}
\renewcommand{\epigraphflush}{center}
\setlength{\epigraphwidth}{.75\textwidth}
%\usepackage{endnotes}
\usepackage{fullpage}

\author{}
\briefauthor{Friend}
\institution{}
\authornote{Submitted for \textsc{COURSE_PREFIX_AND_NUMBER} \\ CLASS_TITLE \\ I._M._A._PROFESSOR, \textsc{phd}, ACADEMIC_RANK_OR_TITLE \\ University of Central Florida}
\title{Handout: Genre Analysis}% use \par to break long titles
\runninghead{Genre Analysis}% this is used in running headers and in pdf metadata
\date{}
% Preamble and document properties (end)

\begin{document} % start of document (fold)
% frontmatter (fold)

\settitle % sets the document title and other relevant information

%\begin{abstract}
%	Place abstract text here...
%\end{abstract}
{\small\noindent Taken from \emph{Scenes of Writing: Strategies for Composing with Genres} by Devitt, Reiff, and Bawarshi (2004), pp.\ 93--94.}

\settitle % sets the title after the title page when applicable
\setlistoftodos% sets list of \todos; automatically disabled when memoir class is called with final option
%\settableofcontents% sets table of contents
%\setlistoftables% sets list of tables
%\setlistoffigures% sets list of figures

% frontmatter (end)

%\section[Guidelines for Analyzing Genres]{Guidelines for Analyzing Genres\footnote{Guidelines from \emph{Scenes of Writing: Strategies for Composing with Genres} by Devitt, Reiff, and Bawarshi (2004), pp.\ 93--94.}} % (fold)
\label{sec:guidelines_for_analyzing_genres}

\section{Collect Samples of the Genre} % (fold)
\label{sub:collect_samples_of_the_genre}
If you are studying a genre that is fairly public, such as the wedding announcement, you can look at samples from various newspapers. You can also locate samples of a genre in textbooks and manuals about the genre. If you are studying a less public genre, such as a Patient Medical History Form, you might have to visit different doctors’ offices to collect samples. Try to gather samples from more than one place so that you get a more accurate picture of the complexity of the genre. The more samples of the genre you collect, the more you will be able to notice patterns within the genre.
% subsection collect_samples_of_the_genre (end)

\section{Identify the Scene and Describe the Situation in which the Genre is Used} % (fold)
\label{sub:identify_the_scene_and_describe_the_situation_in_which_the_genre_is_used}
Try to identify the larger scene in which the genre is used. Seek answers to questions about the genre’s situation such as the ones below:
\begin{compactdesc}
	\item[Setting] Where does the genre appear? How and when is it transmitted and used? With what other genres does this genre interact?
	\item[Subject] What topics, issues, ideas, questions, etc. does the genre address? When people use this genre, what is it that they are they interacting about?
	\item[Participants] Who uses the genre?

\emph{Writers:} Who writes the texts in this genre? Are multiple writers possible? What roles do they perform? What characteristics must writers of this genre possess? Under what circumstances do writers write the genre (e.g., in teams, on a computer, in a rush)?

\emph{Readers:} Who reads the texts in this genre? Is there more than one type of reader for this genre? What roles do they perform? What characteristics must readers of this genre possess? Under what circumstances do readers read the genre (e.g., at their leisure, on the run, in waiting rooms)?
	\item[Purposes] Why do writers write this genre and why do readers read it? What purposes does the genre fulfill for the people who use it?
\end{compactdesc}
% subsection identify_the_scene_and_describe_the_situation_in_which_the_genre_is_used (end)

\section{Identify and Describe Patterns in the Genre’s Features} % (fold)
\label{sub:identify_and_describe_patterns_in_the_genre_s_features}
What recurrent features do the samples share? For example: 
\begin{compactitem}
	\item What \textbf{content} is typically included? What excluded? How is the content treated? What sorts of examples are used? What counts as evidence (personal testimony, facts, etc.)?
	\item What \textbf{rhetorical appeals} are used? What appeals to logos, pathos, and ethos appear? 
	\item How are texts in the genres \textbf{structured}? What are their parts, and how are they organized? 
	\item In what \textbf{format} are texts of this genre presented? What layout or appearance is common? How long is a typical text in this genre? 
	\item What types of \textbf{sentences} do texts in the genre typically use? How long are they? Are they simple or complex, passive or active? Are the sentences varied? Do they share a certain style? 
	\item What \textbf{diction} (types of words) is most common? Is a type of jargon used? Is slang used? How would you describe the writer’s voice?

\end{compactitem}
% subsection identify_and_describe_patterns_in_the_genre_s_features (end)

\section{Analyze What These Patterns Reveal about the Situation and Scene} % (fold)
\label{sub:analyze_what_these_patterns_reveal_about_the_situation_and_scene}
What do these rhetorical patterns reveal about the genre, its situation, and the scene in which it is used? Why are these patterns significant? What can you learn about the actions being performed through the genre by observing its language patterns? What arguments can you make about these patterns? As you consider these questions, focus on the following:

\begin{compactitem}
	\item What do participants have to know or believe to understand or appreciate the genre? 
	\item Who is invited into the genre, and who is excluded? 
	\item What roles for writers and readers does it encourage or discourage? 
	\item What values, beliefs, goals, and assumptions are revealed through the genre’s patterns? 
	\item How is the subject of the genre treated? What content is considered most important? What content (topics or details) is ignored? 
	\item What actions does the genre help make possible? What actions does the genre make difficult? 
	\item What attitude toward readers is implied in the genre? What attitude toward the world is implied in it?
\end{compactitem}
% subsection analyze_what_these_patterns_reveal_about_the_situation_and_scene (end)

% section guidelines_for_analyzing_genres (end)


% backmatter (fold)

%\setappendix*% begins alphabetic section numbering and prints "Appendi(x|ces)"; starred command prints nothing
%\setreferences % typesets the reference list; if no references are used, the list will not be typeset

% backmatter (end)

\end{document} % (end)