\documentclass[10pt,oneside,twocolumn]{amsart}	%defines this as an article
\usepackage{chrisfriend-comp} %provides formatting declarations for page, headers, figures, textcolor, comments, and bibliographic styles
\usepackage{chrisfriend-OTF-support} %provides support for OTF system fonts; incompatible with latex, rtf2latex, & ht4latex
%\usepackage[utf8]{inputenc} %support for smallamp?

%\usepackage{tabularx}
\usepackage{tabulary} % allows for the tables I make rubrics with
%\usepackage{supertabular}
\usepackage{xtab} % allows tables to span pages
\usepackage{booktabs} % allows fancy lines in tables
\usepackage{rotating} % allows landscape tables
\usepackage{lscape} % allows rotated longtables
\usepackage{multirow} % allows rowspanning
\usepackage{enumitem} % helps with the overview
\usepackage{acronym}
\input{acronyms}
\usepackage{paralist}
%\usepackage{draftwatermark}
%\usepackage{multicol}

\title[DC Analysis]{Assignment Sheet: Analysis of Discourse Community Characteristics}
\chead{\scriptsize{\MakeUppercase{DC Analysis}}}
\lhead{\scriptsize{\textsc{enc 1101}}}


\begin{document}
%\bibliographystyle{abbrv}
\thispagestyle{empty}
\setlength{\columnsep}{.25in}
\twocolumn[
%\vspace{-3in}
\begin{center}
\huge
{\includegraphics[height=1.75\baselineskip]{pegasus.pdf}}

\textbf{Assignment Sheet:\\ Analysis of Discourse Community Characteristics}

{\normalsize Chris Friend • \textsc{enc1101} • Spring 2014}

\end{center}
\vspace{\baselineskip}
] %Use for column-spanning the title

%\begin{multicols}{2}
	\section{Background} % (fold)
	\label{sec:background}
		After reading one author's conceptualization of what it takes to make a \emph{discourse community}, your job is to select a group of people you believe meets Swales' six defining characteristics and answer this question: \textbf{How does the group you chose exhibit the characteristics of a discourse community?} Present your conclusions in a paper at least two pages long.
	% section background (end)

	\begin{comment}
		\section{Objectives} % (fold)
		\label{sec:details}
		\todo[inline]{This section needs to be retooled to be aligned to student needs.}
			\begin{itemize}
				\item Understand how language practices mediate group activities
				\item Examine the discourses and texts of different communities
				\item Understand how language plays a role in discourse community enculturation
				\item Identify the relationship between language, identity, and authority
				\item Acquire tools for successfully responding to varied discourse conventions and genres in different classes
			\end{itemize}
		% section details (end)
	\end{comment}

\section{Purpose} % (fold)
\label{sec:purpose}
This assignment will help you understand what a discourse community is and how one of them functions. You will learn to think deeply about the use of written communication within a group, and you will examine how a group functions.
% section purpose (end)

\begin{table}[b]
	\caption{Evaluation Rubric}\label{tab:rubric}
\begin{tabulary}{\columnwidth}{cJ}
	%Human-readable original of content below is in rubrics.numbers file.

\toprule 
\textbf{Great} & 	\textbf{Evaluates} how the combination of characteristics creates a sense of group identity	\\
\midrule \textbf{Good} & 	\textbf{Illustrates} a hierarchy of characteristics within the group	\\
\midrule \textbf{Okay} &	\textbf{Explains}, using examples, how the chosen group meets the characteristics	\\
\midrule \textbf{Poor} &	\textbf{States} that the chosen group is a discourse community in terms of the characteristics		\\
\midrule \textbf{Bad} &	\textbf{Omits} any claim that the chosen group qualifies as a discourse community	\\
	%End of human-readable pasted content.
	\bottomrule
\end{tabulary}
\end{table}
% section rubric (end)

	\section{Required Reading} % (fold)
	\label{sec:readings}
	For this assignment, you will need to read ``The Concept of Discourse Community'' by John Swales, pages 466--80 of your \ac{wawbk} text.
	% section readings (end)

\defaultleftmargin{2em}{}{}{}
	\section{Procedure} % (fold)
	\label{sec:procedure}
		For this assignment, you are  applying the  characteristics identified by Swales on pages 471–73 of your \ac{wawbk} textbook. Show that a specific group you have been a part of is a discourse community.
		\begin{compactitem}
			\item Before you write,
		\begin{compactenum}
			\item brainstorm several groups you are (or have been) in that you think might be considered discourse communities and
			\item consider how those groups use text and language, then choose one group to study. %closely
		\end{compactenum}
		\item In your paper,
		\begin{compactenum}
			\item show how the group you chose possesses each of the six defining characteristics from Swales and
			\item find connections between characteristics to show how they help the group function.
		\end{compactenum}
		\item To make your paper exceptional, also
		\begin{compactenum}
			\item identify which characteristics have greater priority than others, perhaps for group function or for new members seeking membership and
			\item show how the combination of those six characteristics form the group's sense of identity.
		\end{compactenum}
		\end{compactitem}
	% section process (end)

	\section{Evaluation} % (fold)
	\label{sec:rubric}
	Your task for this assignment is to identify a specific group of people as being a discourse community. Your work will be assessed using the evaluation criteria presented in Table~\ref{tab:rubric}.
%\end{multicols}


\begin{comment}
	\section{Formatting} % (fold)
	\label{sec:formatting}
	As with the previous assignments, you are expected to use \textsc{mla} formatting conventions, including:
	\begin{itemize}
		\item double-spaced lines,
		\item one-inch margins on all sides and half-inch indents for paragraphs,
		\item a 12-point typeface with serifs (like Times New Roman, \emph{not} Calibri), and
		\item parenthetical citations, where appropriate, and a Works Cited page.
	\end{itemize}
	% section formatting (end)
\end{comment}

\end{document}
