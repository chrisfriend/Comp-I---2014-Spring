\documentclass[11pt, oneside]{amsart}	%defines this as an article
\usepackage{chrisfriend-comp} %provides formatting declarations for page, headers, figures, textcolor, comments, and bibliographic styles
\usepackage{chrisfriend-OTF-support} %provides support for OTF system fonts; incompatible with latex, rtf2latex, & ht4latex
%\usepackage[utf8]{inputenc} %support for smallamp?

%\usepackage{tabularx}
\usepackage{tabulary} % allows for the tables I make rubrics with
%\usepackage{supertabular}
\usepackage{xtab} % allows tables to span pages
\usepackage{booktabs} % allows fancy lines in tables
\usepackage{rotating} % allows landscape tables
\usepackage{lscape} % allows rotated longtables
\usepackage{multirow} % allows rowspanning
\usepackage{enumitem} % helps with the overview
\usepackage{acronym}
\input{acronyms}
%\usepackage{paralist}
\usepackage{draftwatermark}

\title[Science Accommodation]{Assignment Sheet: Analysis of Science Accommodation}
\chead{\scriptsize{\MakeUppercase{Science Accommodation}}}
\lhead{\scriptsize{\textsc{enc 1101}}}

\begin{document}
%\bibliographystyle{abbrv}
\thispagestyle{empty}
\vspace{-2in}
\begin{center}
\huge
\includegraphics[height=1.75\baselineskip]{pegasus.pdf}

\textbf{Assignment Sheet:\\ Analysis of Science Accommodation}

{\normalsize Chris Friend • \textsc{enc1101} • Spring 2014}
\end{center}
\vspace{1.5\baselineskip}

\section{Background and Purpose} % (fold)
\label{sec:background}
For this assignment, you will explore how authors can create authority within a variety of writing environments. An awareness of authority in writing situations helps as a student in academia as you work to become a member of a discourse community in your major. As a student, you have little authority in your field, but that doesn't mean you can't write as though you do.

For this assignment, you will find a scientific discovery or finding and analyze the differences between its presentation in media (low authority) and in academia (high authority). Comparing the science article with the research you've read in this class, you will see how the academic presentation of scientific ideas differs from the academic presentation of composition studies. When you write for classes in other fields, be sure to see how those fields expect you to think, to write, and to present your ideas.

Your job for this assignment is to locate a popular or non-technical report of a scientific finding, discovery, or announcement. Then find the corresponding academic journal article that reports the original research, compare the two presentations, and draw conclusions about the differences you observe in the discourses. Your ultimate goal is to answer this: \textbf{\emph{How} and \emph{why} do authors make adjustments for different discourse communities?}
% section background (end)


\section{Required Readings} % (fold)
\label{sec:readings}
For this assignment, you will need to read these articles from your \ac{wawbk} text:
\begin{itemize}
	\item Mirabelli, ``Learning to Serve: The Language and Literacy of Food Service Workers'' (pp.\ 538--56)
	\item Penrose and Geisler, ``Reading and Writing Without Authority'' (pp.\ 602--17)
	\item Wardle, ``Identity, Authority, and Learning to Write in New Workplaces'' (pp.\ 520--37)
\end{itemize}
% section readings (end)

\section{Procedure} % (fold)
\label{sec:procedure}
Although more detailed instructions and a menu of potential guiding questions can be found on pages 714–17 of your \ac{wawbk} text, the general stages of the process are these:
\begin{itemize}
	\item Before you write,
\begin{description}
	\item[Find articles] If you start by finding a popular genre that cites its source, finding the academic journal is simple. Browse the \href{http://www.npr.org/sections/science/}{\textsc{npr} science webpage} for an article that interests you. Because \textsc{npr} is good at documenting sources, you should be able to find the name of the academic journal somewhere inside the report. Use \href{http://scholar.google.com}{Google Scholar} or the \href{http://library.ucf.edu}{\textsc{ucf} Library website} to find the original article.
	\item[Analyze the sources] Read both articles, noting differences in authority, claims, lexis, and tone. Answer the questions found on \ac{wawbk} page 715 as a start, to help identify what you do and don't know about the sources. Recall the analysis done in class with the sample articles and repeat the process with the documents you chose.
	\item[Organize your thoughts] Based on your responses to the questions from your text, how could you organize your findings? What categories do you see? Create an organization for your paper based on the results of your analysis; it might follow the columns on the rubric in Table~\ref{tab:rubric}. %From there, determine your structure from the Planning stage.
\end{description}
	\item In your paper,
	\begin{itemize}
		\item Position your thinking with respect to the articles you read in your \ac{wawbk} text. Use Swales' \textsc{cars} model to structure your beginning.
		\item Explain the conclusions you reached through your analysis of the source documents. What do you know about the expectations of each discourse communities based on the writing style of the authors?
	\end{itemize}
	\item To make your paper exceptional,
	\begin{itemize}
		\item Connect your findings with your understanding of the discourse community in which each article exists. How does each kind of writing specifically fit its situation?
		\item Identify the audience expectations addressed by characteristics of your sample articles. What do the consumers of those genres need from their sources?
	\end{itemize}
\end{itemize}
% section process (end)

\begin{table}[b]
	\caption{Analysis of Science Accommodation Grading Rubric}\label{tab:rubric}
\begin{tabulary}{\textwidth}{cJJ}
	\toprule  & \textbf{\textsc{Claims}} (10 pts) & \textbf{\textsc{Authority}} (10 pts) \\ %removed & \textbf{\textsc{Lexis and Tone}} (7 pts) \\ %removed & \textbf{\textsc{Values}}


\midrule \textbf{A} & 	Evaluates the connections among the authors’ claims, the audiences’ expectations, and each discourse community	&	Evaluates why the authors' use of authority is appropriate for each discourse community\\%	&	Evaluates how the lexis and tone used in each source is appropriate for each discourse community	\\
\midrule \textbf{B} & 	Illustrates how each authors’ use of claims fits the purpose of the document 	&	Illustrates how citation and authority works as a form of negotiation\\%	&	Illustrates how the lexis and tone of each source creates or obscures meaning	\\
\midrule \textbf{C} & 	Explains, with examples, each author’s specific decisions for making claims	&	Explains, with examples, styles of citation, quoting, and establishing authority\\%	&	Explains, with examples, differences in authors' tones and chosen lexis	\\
\midrule \textbf{D} & 	States the different claims made by each author	&	States the difference in authority in articles\\%	&	States nature of lexis \& tone presented in each article	\\
\midrule \textbf{F} &	Omits mention of authors' claims	&	Omits reference to issues of authority or sources\\%	&	Omits discussion of lexis/tone used in sources	\\


	\bottomrule
\end{tabulary}
\end{table}
% section rubric (end)

\begin{comment}
\section{Evaluation} % (fold)
\label{sec:rubric}
Your task for this assignment is to analyze how and why academic discourse differs from popular discourses, explaining how authors negotiate authority in different discourse communities. Your work will be assessed using the detailed evaluation criteria presented in Table~\ref{tab:rubric}, but please note the following:
\begin{itemize}
	\item Each assessed category requires interaction with the text. You must refer to the texts you are analyzing in order to make your arguments.
	\item Analysis is a basic requirement of this assignment. Analyzing the articles earns a C. Drawing insightful conclusions based on your analysis is necessary for an A.
\end{itemize}

	\section{Formatting} % (fold)
	\label{sec:formatting}
	As with the previous assignments, you are expected to use \textsc{mla} formatting conventions, including:
	\begin{itemize}
		\item double-spaced lines,
		\item one-inch margins on all sides and half-inch indents for paragraphs,
		\item a 12-point typeface with serifs (like Times New Roman, \emph{not} Calibri), and
		\item parenthetical citations, where appropriate, and a Works Cited page.
	\end{itemize}
	% section formatting (end)
\end{comment}

\end{document}
