\documentclass[10pt, oneside,twocolumn]{amsart}	%defines this as an article
\usepackage{chrisfriend-comp} %provides formatting declarations for page, headers, figures, textcolor, comments, and bibliographic styles
\usepackage{chrisfriend-OTF-support} %provides support for OTF system fonts; incompatible with latex, rtf2latex, & ht4latex
%\usepackage[utf8]{inputenc} %support for smallamp?

%\usepackage{tabularx}
\usepackage{tabulary} % allows for the tables I make rubrics with
%\usepackage{supertabular}
\usepackage{xtab} % allows tables to span pages
\usepackage{booktabs} % allows fancy lines in tables
\usepackage{rotating} % allows landscape tables
\usepackage{lscape} % allows rotated longtables
\usepackage{multirow} % allows rowspanning
\usepackage{enumitem} % helps with the overview
\usepackage{acronym}
\input{acronyms}
%\usepackage{paralist}
\usepackage{draftwatermark}
%\usepackage{fullpage}

\title[Multi-Dimensional Definitions]{Assignment Sheet: Multi-Dimensional Definitions}
\chead{\scriptsize{\MakeUppercase{Multi-Dimensional Definitions}}}
\lhead{\scriptsize{\textsc{enc 1101}}}

\begin{document}
%\bibliographystyle{abbrv}
\thispagestyle{empty}
\setlength{\columnsep}{.125in}
\twocolumn[
%\vspace{-3in}
\begin{center}
\huge
{\includegraphics[height=1.75\baselineskip]{pegasus.pdf}}

\textbf{Assignment Sheet:\\ Multi-Dimensional Definitions}

{\normalsize Chris Friend • \textsc{enc1101} • Spring 2014}
\end{center}
\vspace{\baselineskip}
] %Use for column-spanning the title

\section{Background and Purpose} % (fold)
\label{sec:background}
Articulating and conveying ideas is limited to language, and academics often create new language or appropriate and amend existing language in an effort to construct and share new ideas.  By exploring the varying definitions given to a discipline-specific term, you will gain skills for negotiating specialized language and its role in academic conversations.  You will become more adept at understanding and engaging in (participating in) those academic discussions. For this assignment, you will explore the following question:  \textbf{How does specialized language help experts or academics explain their ideas?}
% section purpose (end)

\section{Required Reading} % (fold)
\label{sec:readings}
No readings from academic articles are needed. However, to begin your work, you will need to find a term or concept with specific academic meaning in an academic text (or from a classroom experience.)  You’ll need to locate and read about this term or concept in both a general dictionary and a discipline-specific dictionary or encyclopedia. 
% section readings (end)

\section{Procedure} % (fold)
\label{sec:procedure}
First, identify a term or concept that has specific academic meaning and would be unfamiliar to a general audience.  Choose your term from either:
\begin{enumerate}
	\item An academic essay we’ve read in class, an academic text in your area of interest, or another academic text you are using for another class 

\item A classroom experience in which you encountered a new term or concept
\end{enumerate}
Then, using specialized reference works, such as discipline-specific dictionaries and encyclopedias, clearly define the concept and explain its significance.  Use examples to show the use or function of the concept in context.

Finally, explain how your new discipline-specific knowledge enhances your understanding of the text or experience that first motivated you.  In addition, excellent papers will explain how defining your term adds to your understanding of the discipline.
% section process (end)

\begin{table}[b]
	\caption{Multi-Dimensional Definitions Grading Rubric}\label{tab:rubric}
\begin{tabulary}{\columnwidth}{cJJ}
	\toprule 
\textbf{A} & 	Evaluates how the term’s function within the passage shows how the discipline uses shared knowledge to construct meaning.	\\
\midrule \textbf{B} & 	Illustrates how an understanding of the specialized definition of the term enhances the meaning of the passage or lecture.	\\
\midrule \textbf{C} &	Explains, with examples, the differences between the term’s generalized and specialized definitions.	\\
\midrule \textbf{D} &	States both the generalized and specialized definitions of the term.	\\
\midrule \textbf{F} &	Omits and discussion of the term’s specialized definition.	\\
	\bottomrule
\end{tabulary}
\end{table}
% section rubric (end)

\section{Evaluation} % (fold)
\label{sec:rubric}
Your task for this assignment is to understand how a discipline-specific academic community uses and/or appropriates a term or concept, and to share how your understanding of the term or concept enhances your understanding of the broader passage, argument or claim where the term or concept is used.  An excellent paper will also note how the use of the term or concept adds to your understanding of the discipline.  Your work will be assessed using the evaluation criteria presented in Table~\ref{tab:rubric}.

\begin{comment}
	\section{Formatting} % (fold)
	\label{sec:formatting}
	As with the previous assignments, you are expected to use \textsc{mla} formatting conventions, including:
	\begin{itemize}
		\item double-spaced lines,
		\item one-inch margins on all sides and half-inch indents for paragraphs,
		\item a 12-point typeface with serifs (like Times New Roman, \emph{not} Calibri), and
		\item parenthetical citations, where appropriate, and a Works Cited page.
	\end{itemize}
	% section formatting (end)
\end{comment}

\end{document}
